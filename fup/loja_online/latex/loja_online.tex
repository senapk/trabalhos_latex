\documentclass[12pt]{article}
 
\usepackage[utf8x]{inputenc}
\usepackage[brazilian]{babel}
\usepackage{fontenc}
\usepackage{graphicx} 
\usepackage{listings}
\usepackage{xcolor}
\usepackage{indentfirst}
\usepackage{pdflscape}
\usepackage[bottom=3cm,top=3cm,left=3cm,right=3cm]{geometry} 
\usepackage[pdftex]{hyperref} %permiti \url

\usepackage{wallpaper}
\usepackage{subfig}

\usepackage{fancyhdr}
\pagestyle{fancy}
\fancyhf{}
\rhead{QXCode}
\lhead{Loja Online}
\fancyfoot[R]{\thepage}
%\rfoot{Page \thepage}

\usepackage[absolute]{textpos}

\lstset{
%    language=java,
    language=c++,
    keywordstyle=\bfseries\ttfamily\color[rgb]{0,0,1},
    identifierstyle=\ttfamily,
    commentstyle=\color[rgb]{0.133,0.545,0.133},
    stringstyle=\ttfamily\color[rgb]{0.627,0.126,0.941},
    showstringspaces=false,
    basicstyle=\small,
    tabsize=2,
    breaklines=true,
    frame=single
}

\renewcommand{\tt}[1]{\lstinline|#1|}
\renewcommand{\bf}[1]{\textbf{#1}}
\newcommand{\code}[1]{\emph{#1}}

\begin{document}

\ThisULCornerWallPaper{1}{./imagens/header}

\begin{textblock}{15}(0.4, 0.4)
\noindent
\begin{center}
\LARGE{\bf{QXCode - Quixadá Coding Team}}\\
\large{\bf{Fundamentos de Programação}} \\
\large{\bf{\today}}
\end{center}
\end{textblock}

\title{\bf{Loja Online\\Aplicativo Gerente e Carrinho de Compras}}

\author{
David Sena \thanks{sena.ufc@gmail.com}, 
Davi Magalhães. \thanks{davi.ipu@hotmail.com}
}

\date{}

\maketitle
\thispagestyle{empty}

%#################################################################
%#################################################################
%#################################################################
%#################################################################

\begin{figure}[h!]
\centering
\includegraphics[width=0.4\linewidth]{imagens/loja_virtual}
\label{fig:loja}
\end{figure}

\section{Instruções Gerais}
O objetivo deste trabalho é simular o sistema de uma loja online. Os gerentes cadastram produtos para serem vendidos na loja e os clientes adicionam esses produtos a seus carrinhos.

Para realizar esse trabalho você precisará lidar com entrada e saída tanto do terminal como de arquivos. Também precisará do conteúdo de vetores e estruturas.

Devem ser gerados dois arquivos executáveis. Um para o gerente e outro para o cliente. A persistência dos dados deve ser feita em arquivos.

\section{Módulo Gerente}
No Módulo Gerente do seu sistema o gerente de vendas da loja deverá ser capaz de:

\begin{enumerate}
  \item Visualizar o estoque da loja
  \item Inserir produto no estoque
  \item Alterar produto do estoque
  \item Remover produto do estoque
  \item Finalizar sessão
\end{enumerate}

Nas informações do produto deverá conter o seu nome, preço \textit{(float)}, se o produto é vendido em \textbf{U}nidade ou \textbf{P}eso e a quantidade em estoque. 

Ao visualizar o estoque da loja, seu sistema deverá gerar automaticamente um identificador para todos os produtos (ID), para que a alteração e/ou remoção do mesmo possa ser feita de modo mais eficiente ao ser buscado pelo sistema.

Ao finalizar a sessão, tudo que tiver sido inserido, alterado ou removido deverá ser salvo num arquivo de chamado estoque.txt.

Uma {\it sugestão} é que você utilize o seguinte formato:
“Nome do produto; preço; unidade ou peso; quantidade em estoque”, separados por ponto-e-vírgula e um produto por linha.

Exemplo:

\begin{verbatim}
	  Coca-cola; 3.6; U; 56
	  Biscoito de Chocolate; 1.5; U; 86
	  Peito de Frango; 14.0; P; 16
\end{verbatim}

Ao executar o programa, automaticamente tudo que estiver em estoque.txt deverá ser carregado em memória, se o arquivo não existir, deverá ser criado.

\subsection{Exemplo}
Exemplo em execução. Os símbolos \verb|>>| significam a entrada de dados do usuário.

\begin{verbatim}
# GERENTE DA LOJA #
H - Help
V - Visualizar o estoque da loja
I - Inserir produto no estoque
A - Alterar produto do estoque
R - Remover produto do estoque
F - Finalizar sessão

Comando >> V

# ESTOQUE DA LOJA #
ID; NOME; PREÇO; UND. OU PESO; QUANTIDADE
1; Coca-cola; 3.6; U; 56
2; Biscoito de Chocolate; 1.5; U; 86
3; Peito de Frango; 14.0; P; 16

Comando >> I

# INSERIR PRODUTO NO ESTOQUE #
Nome do produto >> Tampico
Preço >> 4.5
Tipo >> U
Quantidade >> 34
Produto inserido com sucesso!

Comando: >> A

# ALTERAR PRODUTO EM ESTOQUE #
ID do produto >> 2

# ALTERANDO #
NOME; PREÇO; UND. OU PESO; QUANTIDADE
Biscoito de Chocolate; 1.5; U; 86

Novo nome >> Biscoito de Morango
Novo preço >> 1.6
Novo tipo >> U
Nova quantidade >> 47
Produto alterado com sucesso!

Comando >> V

# ESTOQUE DA LOJA #
ID; NOME; PREÇO; UND. OU PESO; QUANTIDADE
1; Coca-cola; 3.6; U; 56
2; Biscoito de Morango; 1.6; U; 47
3; Peito de Frango; 14.0; P; 16
4; Tampico; 4.5; U; 34

Comando >> H

# GERENTE DA LOJA #
H - Help
V - Visualizar o estoque da loja
I - Inserir produto no estoque
A - Alterar produto do estoque
R - Remover produto do estoque
F - Finalizar sessão

Comando >> F

# Tudo salvo em estoque.txt! #
\end{verbatim}

Para deletar um produto do estoque o sistema só deverá pedir o ID. Deve ser mostrado se a operação foi bem sucedida ou não. Da mesma forma, em alterar, deve ser avisado caso o ID passado não corresponda a nenhum produto.

Você ainda deve estar atendo a outras situações importantes.

\begin{itemize}
\item Não permitir dois produtos com o mesmo nome.
\item Não permitir produtos com preço ou quantidade negativa.
\item Apenas devem ser permitidos os flags Peso e Unidade.
\end{itemize}

\section{Módulo Carrinho}
No Módulo Carrinho do seu sistema funcionará tal qual o carrinho de compras de uma loja virtual. O cliente deverá ser capaz de:

\begin{enumerate}
  \item Visualizar o estoque da loja
  \item Visualizar produtos no carrinho
  \item Inserir produto no carrinho
  \item Alterar quantidade do produto
  \item Remover produto do carrinho
  \item Finalizar compra
\end{enumerate}

Ao inserir um produto no carrinho, o preço correspondente a quantidade inserida deverá ser mostrado, por exemplo, se uma Coca-cola no valor de 3.6 for inserida, deverá constar no carrinho o valor de 3.6, mas se posteriormente a quantidade for alterada, o valor deverá ser incrementado correspondentemente a nova quantidade, exemplo: duas Coca-colas foram inseridas, então o valor mostrado deverá ser 7.2.

Ao iniciar o carrinho, o arquivo estoque.txt deve ser carregado para memória e ao finalizar o carrinho as quantidades em estoque.txt devem ser atualizadas de acordo com a quantidade de produtos comprados.

Atenção a alguns requisitos importantes.
\begin{itemize}
\item Não deve ser possível comprar mais itens do que os disponíveis no estoque
\item Ao visualizar o carrinho, deve ser mostrado o valor total da compra.
\item Trate adequadamente o fato de tentar adicionar duas vezes o mesmo produto ao carrinho.
\item Apenas altere o estoque.txt na finalização da compra.
\end{itemize}

\section{Sugestões}

Como tanto gerente e carrinho fazem funções parecidas, não repita código. Modularize as ações em funções e coloque em bibliotecas para poder serem usadas tanto por gerente como carrinho. 

Atenção, não repita código. Se quando você quiser modificar algo no seu programa, terá que modificar em mais de um lugar no código, provavelmente há alguma coisa errada. Bom trabalho.

Se precisar de ajuda, lembre-se dos professores, bolsistas, monitores e seus amigos. A ajuda pode estar a um botão de distância.

\begin{figure}[h!]
\centering
\includegraphics[width=0.4\linewidth]{./imagens/help}
\end{figure}


\end{document}
