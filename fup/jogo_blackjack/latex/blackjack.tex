\documentclass[12pt]{article}
 
\usepackage[utf8x]{inputenc}
\usepackage[brazilian]{babel}
\usepackage{fontenc}
\usepackage{graphicx} 
\usepackage{listings}
\usepackage{xcolor}
\usepackage{indentfirst}
\usepackage{pdflscape}
\usepackage[bottom=3cm,top=3cm,left=3cm,right=3cm]{geometry} 
\usepackage[pdftex]{hyperref} %permitir \url

\usepackage{wallpaper}
\usepackage{subfig}

\usepackage{fancyhdr}
\pagestyle{fancy}
\fancyhf{}
\rhead{QXCode}
\lhead{Blackjack}
\fancyfoot[R]{\thepage}
%\rfoot{Page \thepage}


\usepackage[absolute]{textpos}


\lstset{
    language=java,
%    language=c++,
    keywordstyle=\bfseries\ttfamily\color[rgb]{0,0,1},
    identifierstyle=\ttfamily,
    commentstyle=\color[rgb]{0.133,0.545,0.133},
    stringstyle=\ttfamily\color[rgb]{0.627,0.126,0.941},
    showstringspaces=false,
    basicstyle=\small,
    tabsize=2,
    breaklines=true,
    frame=single
}



\renewcommand{\tt}[1]{\lstinline|#1|}
\renewcommand{\bf}[1]{\textbf{#1}}


\begin{document}

\ThisULCornerWallPaper{1}{./imagens/header}

\begin{textblock}{15}(0.4, 0.4)
\noindent
\begin{center}
\LARGE{\bf{QXCode - Quixadá Coding Team}}\\
\large{\bf{Fundamentos de Programação}} \\
\large{\bf{\today}}
\end{center}
\end{textblock}

\title{\bf{Blackjack \\ Um Jogo de Cartas Inocente}}

\author{
David Sena \thanks{sena.ufc@gmail.com}, 
Ronildo Oliveira. \thanks{ro.nildooliveira@hotmail.com}
}

\date{}

\maketitle
\thispagestyle{empty}

%#################################################################
%#################################################################
%#################################################################
%#################################################################


\section{Instruções Gerais}
O objetivo desse trabalho é simular o jogo de Blackjack 21 entre um usuário e a mesa. As opções do jogador são simplificadas em relação ao jogo original. 

%A sequencia da figura \ref{fig:blackjack} ganha o jogo.

\begin{figure}[hf]
\centering
\includegraphics[width=0.3\linewidth]{./imagens/blackjack}
\caption{Ganhou}
%\label{fig:blackjack}
\end{figure}

\url{http://pt.blackjack.org/regras/}

\url{http://pt.wikipedia.org/wiki/Blackjack}

Você pode jogar online uma simulação bem parecida com a que vamos implementar para compreender as regras.

\url{http://jogosonline.uol.com.br/blackjack-gentleman-s-bet_43651.html}

\subsection{Regras Básicas}

Todas as cartas com 10 ou menos (2, 3, 4, 5, 6, 7 ,8, 9, 10) têm um valor igual a seu número. Os naipes não influenciam o valor das cartas (o quatro de paus possui o mesmo valor que o quatro de ouros). Por exemplo: se um jogador receber as seguintes cartas: 2, 5 e 8, o valor de todas as três equivalerá a 15.

Os valetes, damas e reis são todos equivalentes a 10. Como as cartas numeradas, naipes diferentes não afetam o valor dessas cartas. Por exemplo: se um jogador receber um valete ou um rei como suas duas primeiras cartas viradas para cima, elas equivalerão a 20, a segunda pontuação mais alta, abaixo apenas do 21.

A última carta, o ás, tem um valor especial no jogo de blackjack. O ás pode valer \bf{1} ou \bf{11}, com o número real dependente do valor mais vantajoso em uma mão específica. Um ás com um rei, rainha ou valete significa 21 pontos e o jogador ganha imediatamente. Se o valor da soma não ultrapassar 21, faça o ás valer 11. Se ultrapassar, vá transformando os Ás em 1 tentando baixar a soma.

\section{Implementação}
Opções do usuário : Pedir Carta (Hit) ou Parar.

O jogador inicia com 2 cartas e a mesa com 1 carta.

O jogador pode pedir quantas cartas quiser até parar
ou estourar o valor de 21.

Se o jogador parar sem estourar, a mesa vai jogar até
vencer o jogador ou estourar. Se a mesa tiver a mesma
quantidade de pontos do jogador ela vence.

Nos trechos que representam a saída do terminal, os símbolos \verb|>>| representam que o programa para e espera a entrada do usuário.

\subsection{Primeira Etapa}

1a Versão. Uma única rodada, um jogador e a mesa. A cada rodada o programa pergunta se o jogador quer para ou continuar. Se quiser continuar, recebe uma carta aleatória. Abaixo, um exemplo de saída.

\begin{verbatim}
Iniciando Rodada:
# Mesa recebe  7 - Total  7 [ 7 ]
# Voce recebe  A - Total 11 [ A ]
# Voce recebe  2 - Total 13 [ A 2 ]
Pedir = 1, Parar = 2 
>> 1

# Voce recebe  3 - Total 16 [ A 2 3 ]
Pedir = 1, Parar = 2 
>> 2

# Mesa recebe  2 - Total  9 [ 7 2 ]
# Mesa recebe  7 - Total 16 [ 7 2 7 ]
# Mesa (16), Voce (16)
Voce perdeu!
\end{verbatim}

Nova execução do programa.

\begin{verbatim}
Iniciando Rodada:
# Mesa recebe  7 - Total  7 [ 7 ]
# Voce recebe  A - Total 11 [ A ]
# Voce recebe  2 - Total 13 [ A 2 ]
Pedir = 1, Parar = 2 
>> 1

# Voce recebe  Q - Total 13 [ A 2 Q ]
Pedir = 1, Parar = 2 
>> 1

# Voce recebe  5 - Total 18 [ A 2 Q 5 ]
Pedir = 1, Parar = 2 
>> 2

# Mesa recebe  8 - Total 15 [ 7 8 ]
# Mesa recebe  7 - Total 22 [ 7 8 7 ]

# Mesa (22), Voce (18)
Voce Ganhou
\end{verbatim}

\subsection{Segunda Etapa}
2a Versão. Várias rodadas, sistemas de apostas, 
limite mínimo e máximo de apostas. Continua a 
execução até parar ou acabar o dinheiro. Se ganhar,
ganha o dobro da aposta. Verifique se a pessoa
tem o dinheiro pra fazer a aposta.

\begin{verbatim}
INSTRUCOES: O mínimo de aposta é 5 e o máximo 100.
Para finalizar digite -1 no valor da aposta.

Rodada 1:
Dinheiro: 100
Digite valor da aposta ou -1 para sair: 25

# Mesa recebe  7 - Total  7 [ 7 ]
# Voce recebe  A - Total 10 [ A ]
# Voce recebe  2 - Total 12 [ A 2 ]
Pedir = 1, Parar = 2 
>> 1

# Voce recebe  3 - Total 16 [ A 2 3 ]
Pedir = 1, Parar = 2 
>> 2

# Mesa recebe  2 - Total  9 [ 7 2 ]
# Mesa recebe  7 - Total 16 [ 7 2 7 ]
# Mesa (16), Voce (16)
Voce perdeu!


Rodada 2:
Dinheiro: 75
Digite valor da aposta ou -1 para sair: 50
# Mesa recebe  7 - Total  7 [ 7 ]
# Voce recebe  A - Total 10 [ A ]
# Voce recebe  2 - Total 12 [ A 2 ]
Pedir = 1, Parar = 2 
>> 1

# Voce recebe  Q - Total 13 [ A 2 Q ]
Pedir = 1, Parar = 2 
>> 1

# Voce recebe  5 - Total 18 [ A 2 Q 5 ]
Pedir = 1, Parar = 2 
>> 2

# Mesa recebe  8 - Total 15 [ 7 8 ]
# Mesa recebe  7 - Total 22 [ 7 8 7 ]

# Mesa (22), Voce (18)
Voce Ganhou

Rodada 3:
Dinheiro: 125
Digite valor da aposta ou -1 para sair: -1
\end{verbatim}


\subsection{Terceira Etapa}
3a Versão. Verifique as entradas de dados
para não aceitar valores inválidos. Verifique em todas as interações com o usuário.

\begin{verbatim}
Rodada 1:
Dinheiro: 100
Digite valor da aposta ou -1 para sair: vinte
Valor inválido.
Digite valor da aposta ou -1 para sair: 300
Valor inválido.
Digite valor da aposta ou -1 para sair: 99

# Mesa recebe  7 - Total  7 [ 7 ]
# Voce recebe  A - Total 10 [ A ]
# Voce recebe  2 - Total 12 [ A 2 ]
Pedir = 1, Parar = 2 
>> 4 batatas
Valor inválido.
Pedir = 1, Parar = 2 
>> -1
Valor inválido.
Pedir = 1, Parar = 2 
>> 2

# Mesa recebe  A - Total 18 [ 7 A ]
# Mesa (18), Voce (12)
Mesa Ganhou
\end{verbatim}
\begin{verbatim}
Rodada 2:
Dinheiro: 1
Digite valor da aposta ou -1 para sair: -2
Valor inválido.
Digite valor da aposta ou -1 para sair: 600
Valor inválido.
Digite valor da aposta ou -1 para sair: -1
\end{verbatim}

\subsection{Desafio}

Se você ainda quer um desafio, que tal implementar
um jogo para múltiplos jogadores?
Pesquise um pouco sobre as regras para múltiplos 
jogadores.
Uma outra possibilidade é adicionar as regras
adicionais como dobrar ou partir as cartas.

Bom Trabalho.

\end{document}
