\documentclass[12pt]{article}
 
\usepackage[utf8x]{inputenc}
\usepackage[brazilian]{babel}
\usepackage{fontenc}
\usepackage{graphicx} 
\usepackage{listings}
\usepackage{xcolor}
\usepackage{indentfirst}
\usepackage{pdflscape}
\usepackage[bottom=3cm,top=3cm,left=3cm,right=3cm]{geometry} 
\usepackage[pdftex]{hyperref} %permiti \url

\usepackage{wallpaper}
\usepackage{subfig}

\usepackage{fancyhdr}
\pagestyle{fancy}
\fancyhf{}
\rhead{QXCode}
\lhead{Jokenpô}
\fancyfoot[R]{\thepage}
%\rfoot{Page \thepage}

\usepackage[absolute]{textpos}

\lstset{
%    language=java,
    language=c++,
    keywordstyle=\bfseries\ttfamily\color[rgb]{0,0,1},
    identifierstyle=\ttfamily,
    commentstyle=\color[rgb]{0.133,0.545,0.133},
    stringstyle=\ttfamily\color[rgb]{0.627,0.126,0.941},
    showstringspaces=false,
    basicstyle=\small,
    tabsize=2,
    breaklines=true,
    frame=single
}

\renewcommand{\tt}[1]{\lstinline|#1|}
\renewcommand{\bf}[1]{\textbf{#1}}
\newcommand{\code}[1]{\emph{#1}}

\begin{document}

\ThisULCornerWallPaper{1}{./imagens/header}

\begin{textblock}{15}(0.4, 0.4)
\noindent
\begin{center}
\LARGE{\bf{QXCode - Quixadá Coding Team}}\\
\large{\bf{Fundamentos de Programação}} \\
\large{\bf{\today}}
\end{center}
\end{textblock}

\title{\bf{Jokenpô\\Pedra, papel e tesoura}}

\author{
David Sena \thanks{sena.ufc@gmail.com}, 
Davi Magalhães. \thanks{davimagal@outlook.com}
}

\date{}

\maketitle
\thispagestyle{empty}

%#################################################################
%#################################################################
%#################################################################
%#################################################################

\begin{figure}[h!]
\centering
\includegraphics[width=0.4\linewidth]{imagens/pedra-papel-tesoura}
\label{fig:jogo}
\end{figure}

\section{Instruções Gerais}
Este trabalho tem como objetivo que você desenvolva habilidades de desenvolvimento de jogos, como contagem de pontos, jogadas aleatórias, rounds e início e fim do game.

Para realizar este trabalho você precisará lidar com entrada e saída de informações e com geração de dados aleatórios.

\section{Jokenpô V1}
O jogo Jokenpô - ou Pedra, Papel e Tesoura - é jogado por dois jogadores em que ambos escolhem, aleatoriamente dentre as três opções, a sua jogada e as exibem ao mesmo tempo vencendo aquele que a opção escolhida ganha a do adversário, ou empatando quando ambos escolhem a mesma opção.
As regras definem as vitórias do seguinte modo:

\begin{enumerate}
  \item \textbf{Pedra} ganha da \textbf{tesoura}
  \item \textbf{Tesoura} ganha do \textbf{papel}
  \item \textbf{Papel} ganha da \textbf{pedra}
\end{enumerate}

O jogo deverá ser entre um jogador e o computador, em que este deverá fazer suas jogadas aleatoriamente após as do jogador, que então serão comparadas, informado quem venceu e incrementado o número de vitórias do atual vitorioso. Cada jogo será dividido em 5 rounds, que deverá ser exibido em qual estará no momento e ao fim o jogo deverá perguntar se o jogador deseja jogar novamente ou sair, sempre exibindo o número de vitórias de cada participante - jogador e computador.

\subsection{Exemplo}
Exemplo em execução. Os símbolos \verb|>>| significam a entrada de dados do usuário.

\begin{verbatim}
# JOKENPÔ #
Você: 0 | PC: 0
Round: 1 / 5

1 - Pedra
2 - Papel
3 - Tesoura
>> 1
Você jogou PEDRA e o PC PAPEL.
O PC ganhou!

# JOKENPÔ #
Você: 0 | PC: 1
Round: 2 / 5

1 - Pedra
2 - Papel
3 - Tesoura
>> 3
Você jogou TESOURA e o PC PAPEL.
Você ganhou!

# JOKENPÔ #
Você: 1 | PC: 1
Round: 3 / 5

1 - Pedra
2 - Papel
3 - Tesoura
>> 2
Você jogou PAPEL e o PC PAPEL.
Ninguém ganhou!

# JOKENPÔ #
Você 1 | PC: 1
Round: 4 / 5

1 - Pedra
2 - Papel
3 - Tesoura
>>

.
.
.

# JOKENPÔ #
Você: 2 | PC: 1
Round: 5 / 5

1 - Pedra
2 - Papel
3 - Tesoura
>> 3
Você jogou TESOURA e o PC PAPEL.
Você ganhou!

PLACAR FINAL:
Você: 3 | PC: 1

JOGAR NOVAMENTE?
1 - Sim
0 - Sair
\end{verbatim}

\section{Jokenpô V2}

\subsection{Pedra, Papel, Tesoura, Lagarto e Spock}
Nesta segunda versão você deverá implementar o modo do jogo mais recente em que duas novas opções foram adicionadas, \textbf{Lagarto e Spock}, conforme as regras clássicas e as adicionais:

\begin{enumerate}
  \item \textbf{Tesoura} ganha do \textbf{papel}
  \item \textbf{Papel} ganha da \textbf{pedra}
  \item \textbf{Pedra} ganha do \textbf{lagarto}
  \item \textbf{Lagarto} ganha do \textbf{Spock}
  \item \textbf{Spock} ganha da \textbf{tesoura}
  \item \textbf{Tesoura} ganha do \textbf{lagarto}
  \item \textbf{Lagarto} ganha do \textbf{papel}
  \item \textbf{Papel} ganha do \textbf{Spock}
  \item \textbf{Spock} ganha da \textbf{pedra}
  \item \textbf{Pedra} ganha da \textbf{tesoura}
\end{enumerate}

\section{Dúvidas}
Se precisar de ajuda, lembre-se dos professores, bolsistas, monitores e seus amigos.
A ajuda pode estar a um botão de distância. :D

\begin{figure}[h!]
\centering
\includegraphics[width=0.4\linewidth]{./imagens/help}
\end{figure}

\end{document}
